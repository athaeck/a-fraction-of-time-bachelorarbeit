\chapter{Fazit}\label{sec:fazit}
In diesem Kapitel werden die Ergebnisse der Arbeit zusammengefasst. Zum Schluss wird ein Ausblick auf die Zukunft des Spiels gegeben.
\section{Zusammenfassung}
In der vorliegenden Arbeit wurde ein verbesserter Prototyp und eine verbesserte Konzeption aus der Grundlage aus dem Gamedesign Workshop 2021/2022 entwickelt. 

% Zunächst wurde dafür in \say{Kapitel \ref{sec:basics}: \nameref{sec:basics}} eine Analyse der Genre ähnlichen Spiele durchgeführt, und die Grundlagen der Kernspielmechanik 
Dafür wurde zunächst in  \say{Kapitel \ref{sec:basics}: \nameref{sec:basics}} eine Analyse von Genre ähnlichen Spielen sowie Spielen mit einer ähnlichen Spielmechanik durchgeführt, welche in der Spielart Bezüge zu diesem Prototyp aufweisen. Außerdem wurden die Grundlagen der Spielmechanik des Prototyps erörtert.

Im weiteren Schritt wurde das Konzept aus dem Gamedesign Workshop in \say{Kapitel \ref{sec:concept}: \nameref{sec:concept}} überarbeitet und in einen passenden Rahmen gesetzt. Das Konzept umfasst nun die wichtigsten Elemente, die das Spiel auf der konzeptionellen Seite haben sollte. Darunter zählen die Hintergrundgeschichte des Spiels, die Welt in die der Spieler eintaucht, die Möglichkeiten, die ihm geboten werden, um der Geschichte in der Welt zu folgen. Sowie Hilfsmittel, Informationen und Rätsel, die eine solche Geschichte mit sich bringen.

In \say{Kapitel \ref{sec:design}: \nameref{sec:design}} wurde eine visuelle Gestaltung der Welt erschaffen, die einen ersten Eindruck des Spiels vermitteln soll. Darunter zählen das Aussehen des Chronologen, erste prototypische Mockups des \ac{UI}s sowie Weltgegenstände, mit denen der Spieler interagieren kann und eine Auswirkung auf weitere Weltgegenstände haben. Im Rahmen der Spielwelt wurden die konzipierten und umgesetzten Rätsel vorgestellt, welche der Spieler lösen darf. Hinzukommen die bereits konzipierten aber noch nicht umgesetzten Rätsel. 

In \say{Kapitel \ref{sec:dev}: \nameref{sec:dev}} wurde die Umsetzung der Spielmechanik vorgestellt. Zunächst wurden verwendete Technologien vorgestellt, welche bei der Entwicklung des Prototyps eine Rolle gespielt haben. Im Anschluss wurden zunächst die technischen Mängel des ursprünglichen Prototyps analysiert und erklärt, welche in der Weiterentwicklung des Prototyps verbessert werden mussten. Das überarbeitete System wurde ebenfalls in seinen Kernelementen vorgestellt und in der Art so angefertigt, dass sie für den Ausblick des Spiels generisch erweiterbar sind.

Abgeschlossen wurde diese Arbeit in \say{Kapitel \ref{sec:test}: \nameref{sec:test}} mit Nutzertests, welche das grundlegende Prinzip und die Umsetzung des Prototyps auf das Verstehen der Kernmechanik überprüft haben. Die dabei gewonnenen Aspekte wurden sogar teilweise im Kapitel der Konzeption und Umsetzung mit ein bedacht. Die restlichen Aspekte werden in der Zukunft der Entwicklung des Spiels miteinbezogen, um den Prototyp zu verbessern. Durch die Durchführung und Auswertung der Nutzertests konnten ebenso die Forschungsfragen zu einem bestimmten Teil beantwortet werden. Die Komplexität der Rätsel kann in einem ansteigendem und vereinzelt absinkenden Grad weiter entwickelt werden.

\section{Ausblick}\label{sec:ausblick}
% Umsetzung von in usertests aufkommenden bugs
% Umsetzung der nur noch konzipierten inhalt
    % zum Beispiel das entdecken sowie das labor
% Ingame Zwischensequezne zussätzlich
% Dialog system
% Integration von DOTS
% weitere Level
% art stil wechsel ähnlich zu ashen um texturielle unterscheidungen zu intgerieren (stylized / textured low poly)
% einhergehen mit wechsel zu hrp um optisch noch mehr im postprocsssing rauszuholen und zu integrieren
% eine weiter entwicklung bis später mal vermarktung 