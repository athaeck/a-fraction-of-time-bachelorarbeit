\chapter{Grundlagen}\label{sec:basics}
Im folgenden Kapitel werden die für das Spiel wichtigen grundlegenden Elemente beschrieben. Zunächst erfolgt eine erklärende Einführung in das Genre der Adventure-Games, die für die Konzeption dieses Spieles relevant ist. Im Anschluss wird darauf eingegangen, welche Spiele aus dem Adventure oder Action-Adventure-Genre im Single – oder Multiplayer-Modus einen Bezug auf das hier beschriebene Spiel besitzen. Darauffolgend wird auf die für das Spiel wichtige Spielmechaniken eingegangen. Die Betrachtung von Replay-Systemen und der damit einhergehende Determinismus der Physik-Engine ist dabei der wichtigste Bezugspunkt.
\section{Forschungsstand zu Adventure-Games}
Folgendes Zitat beschreibt den Aufbau von Adventure-Games:

\say{\emph{Adventure games focus on puzzle solving within a narrative framework, generally with few or no action elements [...].}} (vgl. \cite{noauthor_what_2012})

Auf dem Spielemarkt befinden sich viele Kooperative-Spiele, die entweder im Offline oder Online-Multiplayer gespielt werden können. Darunter fallen Titel wie \say{\emph{A Way Out}} aus dem Hause \say{\emph{\ac{EA}}} (vgl. \cite{arts_way_2017}) oder \say{\emph{It Takes Two}}, welches ebenfalls aus dem Hause \say{\emph{\ac{EA}}} stammt und von dem Entwicklerstudio \say{\emph{Hazelight}} entwickelt wurde (vgl. \cite{arts_entdecke_2022}). Bekannte Spiele, welche kooperativ im Singleplayer wirken, sind Spiele \say{\emph{The last of us}} oder der \say{\emph{Uncharted}}- Reihe, bei denen an bestimmten Stellen der dargebotenen Rätsel NPC zur Seite springen und dem Spieler helfen (vgl. \cite{noauthor_last_tlou}) (vgl. \cite{noauthor_uncharted_nodate}). Um auf die zeitliche Komponente einzugehen, befassen sich Spiele wie \say{\emph{Quantumbreak}} von \say{\emph{Xbox Games Studio}} (vgl. \cite{noauthor_kaufen_nodate}) oder \say{\emph{Bioshok Infinitie}} (vgl. \cite{noauthor_bioshock_nodate}) von \say{\emph{Irrational Games}} mit geschichtlich festgelegten Zeitreisen bzw. Zeitsprünge in die Vergangenheit. Das bedeutet, dass Reisen in der Zeit als Kernpunkt der Geschichte anzusehen ist und passiert auf lineare Weise. Es hat keine Auswirkung auf das Spielgeschehen. Hingegen beschäftigt sich das Spiel \say{\emph{Life is Strange}} (vgl. \cite{noauthor_life_nodate}) damit, Zeitpunkte in der Vergangenheit zu ändern, um zukünftige Ereignisse zu beeinflussen.

\section{Forschungsstand Replay-Systeme in Videospielen}
Replay oder Wiederholungssysteme gibt es hauptsächlich bei Ego-Shootern und Rennspielen. Bei Ego-Shootern wird dieses System häufig bei der Wiederholung des letzten \say{Kills} in der Runde verwendet. Dabei werden die gespeicherten Inputs der Spieler neu in der Game-Engine ausgewertet, wodurch der Eindruck entsteht, eine Kamera würde das Gesehene aufnehmen (vgl. \cite{noauthor_new_nodate}). Eine solche Kill-Cam ist in dem Spiel \say{\emph{Call of Duty: Black Ops 2}} von \say{\emph{Activision}} (vgl. \cite{noauthor_call_nodate}) integriert. Allerdings sind dies nur die Inputs der Spieler, die in einer deterministischen Engine ausgewertet wurden. Etwas Ähnliches gibt es auch bei Rennspielen, bei denen ausgewählte Rennen nochmals angesehen werden können. In \say{\emph{Forza Horizon 4}} von den Entwicklerstudios \say{\emph{Playground Games}} und \say{\emph{Turn 10 Studios}} ist ein Replay-System eingebaut. Der Spieler kann einzelne Fahrten aufnehmen und diese über die Wiederholfunktion anschauen (vgl. \cite{noauthor_forza_nodate}). Dabei werden auch wieder die Inputs des Spielers gespeichert und beim Betrachten der Wiederholung erneut in der Sequenz ausgespielt. Wie bereits erwähnt, ist es wichtig, dass die Engine deterministisch ist, damit das Endergebnis der nachgeholten Simulation identisch ist zu dem ist, was der Spieler zur Laufzeit des Rennens erlebt hat. 
Das Spiel \say{\emph{The Talos Principle}} vom Entwicklungsteam \say{\emph{Croatem}} ist ebenfalls ein Rätselspiel, bei dem der Spieler bestimmte Bewegungen und Interaktionen aufnehmen kann. Die aufgenommene Bewegung hat ebenfalls einen Einfluss in die Spielwelt wie der Spieler. Der Spieler kann diese Aufnahmen allerdings nicht verschachteln, wie es in dieser Bachelorarbeit der Fall sein wird (vgl. \cite{croteam_talos_2014}).

\section{Determinismus}
Determinismus in Computerspielen bezieht sich auf die Tatsache, dass jede Aktion des Spielers, die in einem bestimmten Zustand des Spiels ausgeführt wird, immer das gleiche Ergebnis hat. Das bedeutet, dass das Spielverhalten vorhersehbar und reproduzierbar ist, da jede Aktion des Spielers die gleiche Auswirkung auf den Zustand des Spiels hat (vgl. \cite{noauthor_game_nodate}). Wenn der Protagonist im Kontext des Spiels von einer Plattform springt und auf dem Boden landet, so sollte es immer denselben Sprung und denselben Fall geben und er sollte an derselben Stelle aufkommen.
Diese Gegebenheit ist wichtig, um ein Replay-System einzubauen, welches den Kern dieses Spiels ausmacht. Deshalb ist es wichtig zu wissen, ob und auf welche Weisen die Engine Unity deterministisch ist. 
Unity ist grundlegend nicht deterministisch. Das liegt daran, dass die Prozesse, die im Hintergrund der Engine laufen, nicht immer gleich ausgeführt werden können, sei es plötzliches Schwanken der Bildrate oder der ungenauen Berechnung der Gravitation. Deshalb ist es notwendig, die Auswirkungen, die es auf den Spielercharakter geben kann, bei der Aufnahme zu berücksichtigen. Es ist allerdings möglich, das System so umzubauen, damit es hinreichend deterministisch ist. In dieser Arbeit wird versucht, ein hinreichend deterministisches System einzubauen.
Da Unity grundlegend nicht deterministisch ist, kann ein Replay-System, wie wir es von Rennspielen oder Ego-Shootern kennen, nicht eingebaut werden. Das Neuinterpretieren der Spieler-Eingaben könnte zu einem anderen Ergebnis führen, als es bei der Aufnahme der Fall war.

Seit März 2022 ist ein erster Experimental-Release eines neuen Technologie-Stacks von Unity erschienen, welcher größere skalierte Simulationen und Performance Scaling unterstützt. Dieser Technologie Stack wird \ac{DOTS} genannt und wird vermutlich in nächster Zeit vollständig erscheinen. Da dieser Stack noch nicht vollständig erschienenen ist, und die Gefahr zu groß ist, dass der Prototyp durch Softwarefehler von \ac{DOTS} nicht spielbar wird, wurde darauf verzichtet, diesen Technologie Stack zu integrieren. Für die weitere Entwicklung dieses Technologie Stacks ist es aber denkbar, ihn einzubinden  (vgl. \cite{technologies_dots_nodate}). Dieser Aspekt wird in \say{Kapitel \ref{sec:ausblick}: \nameref{sec:ausblick}} dieser Thesis erneut aufgegriffen.
